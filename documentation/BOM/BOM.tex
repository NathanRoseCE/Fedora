\documentclass[11pt]{article}
\usepackage{fp}
\usepackage{booktabs}
\usepackage{ragged2e}
\usepackage{longtable}
\newcounter{cnt}
\usepackage{hyperref}
\usepackage{geometry}
\geometry{letterpaper, margin=0.75in}

\title{Fedora BOM}
\author{Nathan Rose}
\date{\today}

\hypersetup{
  colorlinks=true,
  linkcolor=blue,
  urlcolor=blue,
  pdfpagemode=FullScreen,
}

\setcounter{cnt}{0}
\def\inc{\stepcounter{cnt}\thecnt}
\gdef\TotalHT{0}
\newcommand{\product}[5]{%
\inc &#1 &\href{#5}{#4} &#2 &#3 &\FPmul\temp{#2}{#3}\FPround\temp{\temp}{2}\temp 
%% Totalize
\FPadd\total{\TotalHT}{\temp}%
\FPround\total{\total}{2}%
\global\let\TotalHT\total%
\\ }
\newcommand{\totalttc}{
  \TotalHT  }

\begin{document}
\maketitle

\section{BOM}
\begin{longtable}{cp{4.2cm}rrrr}
\toprule
Item & Description & Seller & Price & Qty & Total \\
\midrule
\product{Pi4 4Gb}{55.00}{1}{Pi Shop}{https://www.pishop.us/product/raspberry-pi-4-model-b-4gb/?src=raspberrypi}
\product{Pi4 4Gb}{55.00}{1}{Cana Kit}{https://www.canakit.com/raspberry-pi-4-4gb.html}
\product{Pi4 4Gb}{55.00}{1}{CED}{https://chicagodist.com/products/raspberry-pi-4-model-b-4gb?src=raspberrypi}
\product{Power Supply}{8.00}{3}{Pi Shop}{https://www.pishop.us/product/usb-c-power-supply-5-1v-3-0a-ul-listed-on-off-switch/}
\product{Ethernet Switch}{13.50}{1}{Amazon}{https://www.amazon.com/NETGEAR-5-Port-Gigabit-Ethernet-Unmanaged/dp/B07S98YLHM/ref=asc_df_B07S98YLHM/?tag=\&linkCode=df0\&hvadid=366315306136\&hvpos=\&hvnetw=g\&hvrand=16681842210923765718\&hvpone=\&hvptwo=\&hvqmt=\&hvdev=c\&hvdvcmdl=\&hvlocint=\&hvlocphy=9011497\&hvtargid=pla-792849707490\&ref=\&adgrpid=75347436639\&th=1}
\midrule
    &&&&Total:& \totalttc\\
\bottomrule
\end{longtable}

\section{Justifications}
\subsection{Rasberry Pi4}
The Rasberry pi offers a quick and painless method of testing the software on an ARM processor. This is necessary
as the main processor for Katalyst Space Technologies is an ARM and in order to create a representative test, I
would like to use the same processor family. That being stated the Pi2 v1.1 would be ideal due to the specific
core type they use(Arm A7) is very similar to the A9 likely to be used by Katalyst, however due to the global
chip shortage I could not locate any by a reputable supplier. The only Pi I could find by reputable suppliers
were the Pi4b, which are still ARM A line so the test will be representative just not as close as initially
planned.

A rasberry Pi also supports lower level IO such as UART that I can use to demonstrate that this works over
UART as well as ethernet.

A note too that the three sellers are used as all of these sellers have a 1 pi per customer restriction and
3 is the minimum I need to ensure my software works.

\subsection{Ethernet switch}
I need a small ethernet switch in order to test these messages sent over ethernet. I have enough annoyingly short
ethernet cables in my room that I am MORE THAN HAPPY to donate to my project.

\end{document}



